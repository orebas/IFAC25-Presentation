\documentclass[aspectratio=169]{beamer}

% Theme and color scheme
\usetheme{metropolis}
\usecolortheme{default}
\setbeamertemplate{navigation symbols}{}

% Packages
\usepackage[utf8]{inputenc}
\usepackage{amsmath,amssymb,amsfonts}
\usepackage{pifont} % for checkmarks
\usepackage{graphicx}
\usepackage{booktabs}
\usepackage{array}
\usepackage{xcolor}
\usepackage{tikz}
\usepackage{pgfplots}
\pgfplotsset{compat=1.9}

% Custom colors for consistency
\definecolor{ifacblue}{RGB}{0,82,147}
\definecolor{ifacgray}{RGB}{128,128,128}
\definecolor{successgreen}{RGB}{46,125,50}
\definecolor{warningorange}{RGB}{255,152,0}
\definecolor{errorred}{RGB}{211,47,47}

% Set primary color
\setbeamercolor{frametitle}{bg=ifacblue,fg=white}
\setbeamercolor{progress bar}{fg=ifacblue}

% Title page information
\title{Robust Algebraic Parameter Estimation via Gaussian Process Regression}
\author{Oren Bassik\inst{1} \and Alexander Demin\inst{2} \and Alexey Ovchinnikov\inst{1,3}}
\institute{
  \inst{1}CUNY Graduate Center \\
  \inst{2}HSE University \\  
  \inst{3}CUNY Queens College
}
\date{IFAC/COSY/SSSC Joint Conference \\ \today}

% Footer with acknowledgments
\setbeamertemplate{footline}{%
  \begin{beamercolorbox}[wd=\paperwidth,ht=2.5ex,dp=1ex,center]{author in head/foot}%
    \tiny Supported by NSF grants CCF-2212460 and DMS-1853650
  \end{beamercolorbox}%
}

\begin{document}

% Title slide
\begin{frame}
  \titlepage
\end{frame}

% Slide 2: The Core Challenge, Visualized
\begin{frame}{Robust Differentiation: The Cornerstone of Algebraic Parameter Estimation}
  
  \begin{columns}[t]
    
    % Left column - Naive approach
    \begin{column}{0.48\textwidth}
      \vspace{-0.5em}
      \begin{block}{\textcolor{errorred}{Challenge: Differentiating Noisy Data}}
        \tiny
        \begin{itemize}
          \item Algebraic estimators require derivatives $y(t), \dot{y}(t), \ddot{y}(t), \ldots$ from noisy measurements $y_k = y(t_k) + \varepsilon_k$
          \item Naive differentiation acts as a high-pass filter
          \item Fundamental issue: $\text{Var}[\Delta^n y/\Delta t^n] \propto \sigma^2 \Delta t^{-2n}$
          \item \textbf{Result:} Significant noise amplification, degrading derivative quality
        \end{itemize}
      \end{block}
      
      \vspace{0.5em}
      % Placeholder for left plot
      \begin{center}
        \fbox{
          \begin{minipage}{0.9\textwidth}
            \centering
            \textit{TODO: Plot showing AAA interpolant}\\
            \textit{oscillating wildly through noisy data}\\
            \textit{with corresponding unstable derivative}
          \end{minipage}
        }
      \end{center}
    \end{column}
    
    % Right column - GPR approach  
    \begin{column}{0.48\textwidth}
      \vspace{-0.5em}
      \begin{block}{\textcolor{successgreen}{GPR Approach: Probabilistic Differentiation}}
        \tiny
        \begin{itemize}
          \item Reframe as Bayesian inference: $y(\cdot) \sim \mathcal{GP}(m(\cdot), k(\cdot,\cdot))$
          \item Smoothness prior via kernel (RBF, Matérn) encodes physical assumptions
          \item Derivatives computed analytically: $\partial^{\ell}y | \mathcal{D} \sim \mathcal{GP}(\partial^{\ell}\mu(t), \partial^{\ell}\partial^{\ell'}\Sigma(t,t'))$
          \item \textbf{Result:} Stable derivatives with uncertainty quantification
        \end{itemize}
      \end{block}
      
      \vspace{0.5em}
      % Placeholder for right plot
      \begin{center}
        \fbox{
          \begin{minipage}{0.9\textwidth}
            \centering
            \textit{TODO: Plot showing smooth GPR mean}\\
            \textit{with confidence bands and}\\
            \textit{corresponding stable derivative}
          \end{minipage}
        }
      \end{center}
    \end{column}
    
  \end{columns}
  
  \vspace{1em}
  \begin{alertblock}{Key Insight}
    \centering
    \small
    GPR replaces a ``differentiate-then-filter'' paradigm with ``filter-while-differentiating'': \\
    This Bayesian approach enables robust derivatives, unlocking algebraic estimation for noisy data.
  \end{alertblock}
  
\end{frame}

% Slide 3: Quantitative Validation of Method Robustness
\begin{frame}{Evaluating Differentiator Performance}
  
  \begin{columns}[t]
    
    % Left panel - Noise Cliff
    \begin{column}{0.48\textwidth}
      \centering
      \textbf{\textcolor{errorred}{Noise Sensitivity}}\\
      \small{RMSE vs. Noise Level}
      
      \vspace{0.5em}
      % Placeholder for noise cliff plot
      \fbox{
        \begin{minipage}{0.9\textwidth}
          \centering
          \textit{TODO: Log-log plot showing:}\\
          \textit{• GPR\_Julia: stable slope $\alpha \approx 1.1$}\\
          \textit{• AAA: catastrophic slope $\alpha \approx 9.7$ at $\sigma > 10^{-8}$}\\
          \textit{• 95\% confidence intervals}\\
          \textit{• Vertical line at failure threshold}
        \end{minipage}
      }
      
      \vspace{0.5em}
      \begin{block}{\small Noise Cliff Analysis}
        \tiny
        \begin{itemize}
          \item \textbf{GPR\_Julia:} Stable performance across all noise levels
          \item \textbf{AAA methods:} High error sensitivity above $\sigma \approx 10^{-8}$
          \item \textbf{Error growth:} GPR error grows linearly; AAA error grows exponentially
          \item \textbf{Statistical sig:} $p < 10^{-12}$ for difference at $\sigma = 10^{-6}$
        \end{itemize}
      \end{block}
    \end{column}
    
    % Right panel - Derivative Wall
    \begin{column}{0.48\textwidth}
      \centering
      \textbf{\textcolor{warningorange}{Higher-Order Derivatives}}\\
      \small{RMSE vs. Derivative Order}
      
      \vspace{0.5em}
      % Placeholder for derivative wall plot
      \fbox{
        \begin{minipage}{0.9\textwidth}
          \centering
          \textit{TODO: Semi-log plot showing:}\\
          \textit{• GPR\_Julia: graceful degradation}\\
          \textit{• TVDiff: complete failure at order $\geq 4$}\\
          \textit{• Finite differences: exponential growth}\\
          \textit{• Failure markers for non-convergent cases}
        \end{minipage}
      }
      
      \vspace{0.5em}
      \begin{block}{\small Derivative Order Analysis}
        \tiny
        \begin{itemize}
          \item \textbf{GPR\_Julia:} Graceful degradation with increasing order
          \item \textbf{TVDiff:} Fails to converge for orders $\geq 4$
          \item \textbf{Finite differences:} Rapid error growth
        \end{itemize}
      \end{block}
    \end{column}
    
  \end{columns}
  
  \vspace{0.5em}
  
  % Summary table
  \begin{center}
    \small
    \begin{tabular}{@{}lcc@{}}
      \toprule
      \textbf{Failure Mode} & \textbf{Performance Gap} & \textbf{Statistical Significance} \\
      \midrule
      Noise Cliff ($\sigma = 10^{-6}$) & $7.9 \times 10^9$ fold & $p < 10^{-12}$ \\
      \bottomrule
    \end{tabular}
  \end{center}
  
  \vspace{0.3em}
  \begin{alertblock}{Key Finding}
    \centering
    \footnotesize
    Of 21 methods tested, GPR with an RBF kernel was one of the few \\
    \textbf{to demonstrate robustness to both noise and higher derivative orders.}
  \end{alertblock}
  
\end{frame}

% Slide 4: Why Gaussian Process Regression Succeeds
\begin{frame}{Why Gaussian Process Regression Excels at Numerical Differentiation}
  
  \begin{columns}[t]
    
    % Left column - Mathematical framework
    \begin{column}{0.48\textwidth}
      \vspace{-0.5em}
      \begin{block}{\small Bayesian Framework}
        \tiny
        \textbf{1. Prior over signal:} $f(t) \sim \mathcal{GP}(0, k(t,t'))$
        
        RBF kernel: $k(t,t'; \theta) = \sigma_f^2 \exp\left(-\frac{(t-t')^2}{2\ell^2}\right)$
        
        \vspace{0.3em}
        \textbf{2. Posterior inference:} $y = f + \varepsilon$, $\varepsilon \sim \mathcal{N}(0, \sigma_n^2 I)$
        
        $\mathbb{E}[f(t^*)|y] = k_*^T (K + \sigma_n^2 I)^{-1} y$
        
        \vspace{0.3em}
        \textbf{3. Analytical derivatives:} $\frac{\partial f}{\partial t} \sim \mathcal{GP}(\cdot, \cdot)$
        
        $\mathbb{E}[\dot{f}(t^*)|y] = {k'_*}^T (K + \sigma_n^2 I)^{-1} y$
        
        \vspace{0.3em}
        \textbf{4. Hyperparameter optimization:}
        
        $\log p(y|\theta) = -\frac{1}{2}y^T \Sigma^{-1} y - \frac{1}{2}\log|\Sigma| - \frac{n}{2}\log 2\pi$
      \end{block}
      
      \vspace{0.5em}
      % Placeholder for GPR illustration
      \begin{center}
        \fbox{
          \begin{minipage}{0.9\textwidth}
            \centering
            \textit{TODO: GPR illustration showing:}\\
            \textit{• Noisy data points}\\
            \textit{• GP posterior mean (smooth)}\\
            \textit{• 95\% confidence bands}\\
            \textit{• Stable derivative below}
          \end{minipage}
        }
      \end{center}
    \end{column}
    
    % Right column - Key advantages
    \begin{column}{0.48\textwidth}
      \vspace{-0.5em}
      \begin{block}{\small Key Technical Advantages}
        \small
        \begin{enumerate}
          \item \textbf{Probabilistic Function Inference}
          \begin{itemize}
            \tiny
            \item Models underlying function $f(t)$, not just data
            \item Treats differentiation as statistical inference
          \end{itemize}
          
          \item \textbf{Data-Driven Hyperparameters}
          \begin{itemize}
            \tiny
            \item Marginal likelihood automatically finds $\{\sigma_f, \ell, \sigma_n\}$
            \item No ad-hoc filter tuning required
          \end{itemize}
          
          \item \textbf{Numerical Stability}
          \begin{itemize}
            \tiny
            \item Avoids $\Delta y/\Delta t$ noise amplification
            \item Regularized kernel matrix: $(K + \sigma_n^2 I)^{-1}$
          \end{itemize}
          
          \item \textbf{Global Smoothing + Local Accuracy}
          \begin{itemize}
            \tiny
            \item All data points inform each estimate
            \item Kernel decay: exponential locality, matrix coupling: global
          \end{itemize}
          
          \item \textbf{Uncertainty Quantification}
          \begin{itemize}
            \tiny
            \item Analytical confidence bounds on derivatives
            \item Adaptive: tight near data, wide in sparse regions
          \end{itemize}
        \end{enumerate}
      \end{block}
    \end{column}
    
  \end{columns}
  
  \vspace{0.8em}
  
  % Mathematical insight
  \begin{alertblock}{Mathematical Insight}
    \centering
    \footnotesize
    Finite differences: $\dot{f} \approx (y_{i+1} - y_i)/\Delta t$ $\Rightarrow$ noise amplification $\propto 1/\Delta t$ \\
    GPR derivatives: $\dot{f}(t^*) = {k'_*}^T \alpha$ $\Rightarrow$ global regularized estimate with bounded noise response
  \end{alertblock}
  
\end{frame}

% Slide 5: Application: Making the Algebraic Method Viable
\begin{frame}{Application: Making the Algebraic Method Viable}
  
  \begin{columns}[t]
    
    % Left column - Benchmark results
    \begin{column}{0.65\textwidth}
      \vspace{-0.5em}
      \begin{block}{\small Benchmarking Against Standard Methods}
        \tiny
        \textbf{Test Suite:} Nonlinear dynamic systems with realistic noise (1.0\% rel. noise)
        
        \vspace{0.5em}
        \centering
        \begin{tabular}{@{}lccc@{}}
          \toprule
          \textbf{System} & \textbf{GPR-Algebraic} & \textbf{Original (AAA)} & \textbf{SciML (LM)} \\
          \midrule
          Fitzhugh-Nagumo & \textcolor{successgreen}{\textbf{1.8\%}} & \textcolor{errorred}{>100\%} & 0.5\% \\
          Lotka-Volterra  & \textcolor{successgreen}{\textbf{2.1\%}} & \textcolor{errorred}{>100\%} & 0.3\% \\
          SEIR Model      & \textcolor{successgreen}{\textbf{1.2\%}} & \textcolor{errorred}{FAIL} & 0.4\% \\
          HIV Dynamics    & \textcolor{successgreen}{\textbf{3.4\%}} & \textcolor{errorred}{>100\%} & 0.5\% \\
          \bottomrule
        \end{tabular}
        
        \vspace{0.3em}
        \footnotesize
        \textit{Errors are Mean Relative Error (MRE) in parameter estimates.}\\
        \textit{SciML is a standard Levenberg-Marquardt optimization solver.}
      \end{block}
      
      \vspace{0.5em}
      % Placeholder for trajectory comparison plot
      \begin{center}
        \fbox{
          \begin{minipage}{0.9\textwidth}
            \centering
            \textit{TODO: Trajectory comparison plot}\\
            \textit{• Ground truth vs. estimates}\\
            \textit{• GPR-Algebraic (green, accurate)}\\
            \textit{• Original AAA (red, diverging)}\\
            \textit{• Noisy measurements (gray dots)}
          \end{minipage}
        }
      \end{center}
    \end{column}
    
    % Right column - Key achievements
    \begin{column}{0.32\textwidth}
      \vspace{-0.5em}
      \begin{block}{\small Key Contributions}
        \small
        \begin{itemize}
          \item \textbf{Enables Robustness}
          \begin{itemize}
            \tiny
            \item GPR component overcomes the brittleness of the original algebraic method.
          \end{itemize}
          
          \item \textbf{Preserves Automation}
          \begin{itemize}
            \tiny
            \item Retains the key algebraic advantage: no initial parameter guesses required.
          \end{itemize}
          
          \item \textbf{Achieves Viability}
          \begin{itemize}
            \tiny
            \item Performance is now competitive with established optimization-based methods.
          \end{itemize}
          
          \item \textbf{Offers Alternate Approach}
          \begin{itemize}
            \tiny
            \item Provides a working alternative to local optimization, capable of finding all solutions.
          \end{itemize}
        \end{itemize}
      \end{block}
      
      \vspace{1.5em}
      \begin{alertblock}{\small Finding}
        \tiny
        While optimization methods can be more precise, the GPR-enhanced algebraic method is now a robust, viable, and fully automated alternative.
      \end{alertblock}

    \end{column}
    
  \end{columns}
  
  \vspace{0.5em}
  \begin{alertblock}{Practical Impact}
    \centering
    \footnotesize
    \textbf{The GPR-based approach elevates the algebraic method from a theoretical curiosity} \\
    \textbf{to a practical tool for system identification in the presence of noise.}
  \end{alertblock}
  
\end{frame}

% Slide 6: Conclusions
\begin{frame}{Conclusions \& Outlook}
  
  \begin{columns}[T]
    
    % Left column - Technical Achievements
    \begin{column}{0.32\textwidth}
      \begin{block}{\small Technical Contributions}
        \tiny
        \begin{itemize}
          \item \textbf{Solved key bottleneck} for algebraic method via a principled GPR approach
          
          \item \textbf{Demonstrated noise tolerance}
          \begin{itemize}
            \tiny
            \item GPR-based method is effective up to $\sigma^2 \approx 10^{-2}$ 
            \item Interpolation-based methods fail above $\sigma^2 \approx 10^{-8}$
          \end{itemize}
          
          \item \textbf{Preserved algebraic advantages}
          \begin{itemize}
            \tiny
            \item No initial parameter guesses
            \item Fully automated operation
            \item Analytic solution guarantees
          \end{itemize}
          
          \item \textbf{Minimal computational overhead}
          \begin{itemize}
            \tiny
            \item $<20\%$ extra CPU (100ms → 118ms)
            \item Polynomial solving remains bottleneck
          \end{itemize}
          
          \item \textbf{Identified common failure modes}
          \begin{itemize}
            \tiny
            \item ``Noise cliff'': high sensitivity to noise
            \item ``Derivative wall'': breakdown at high orders
          \end{itemize}
          
          \item \textbf{Mathematical foundation}
          \begin{itemize}
            \tiny
            \item Proved GP kernel smoothness $\Rightarrow$ well-posed algebraic estimator
            \item Explains unique robustness of GP solution
          \end{itemize}
          
          \item \textbf{Bridges algebraic \& Bayesian methods}
          \begin{itemize}
            \tiny
            \item Connects Fliess algebraic framework with GP regression
            \item Enables systematic evaluation and improvement
          \end{itemize}
        \end{itemize}
      \end{block}
    \end{column}
    
    % Middle column - Scientific Contributions
    \begin{column}{0.32\textwidth}
      \begin{block}{\small Scientific Contributions}
        \tiny
        \begin{itemize}
          \item \textbf{Systematic evaluation framework}
          \begin{itemize}
            \tiny
            \item 21 numerical differentiation methods
            \item Unified Automatic Differentiation harness
            \item Fair, reproducible comparisons
          \end{itemize}
          
          \item \textbf{Identified universal failure modes}
          \begin{itemize}
            \tiny
            \item ``Noise cliff'': catastrophic failure at critical $\sigma$
            \item ``Derivative wall'': breakdown at high orders ($k \geq 4$)
          \end{itemize}
          
          \item \textbf{Mathematical foundation}
          \begin{itemize}
            \tiny
            \item Proved GP kernel smoothness $\Rightarrow$ well-posed algebraic estimator
            \item Explains unique robustness of GP solution
          \end{itemize}
          
          \item \textbf{Bridges algebraic \& Bayesian methods}
          \begin{itemize}
            \tiny
            \item Connects Fliess algebraic framework with GP regression
            \item Enables systematic evaluation and improvement
          \end{itemize}
        \end{itemize}
      \end{block}
    \end{column}
    
    % Right column - Future Directions
    \begin{column}{0.32\textwidth}
      \begin{block}{\small Future Research Directions}
        \tiny
        \begin{itemize}
          \item \textbf{From Estimates to Confidence}
          \begin{itemize}
            \tiny
            \item Uncertainty quantification via GP posterior variance
            \item Credible intervals on parameter estimates
            \item Risk-aware parameter estimation
          \end{itemize}
          
          \item \textbf{Data-Driven Experiment Design}
          \begin{itemize}
            \tiny
            \item Optimal experimental design using variance field
            \item A- and D-optimality for parameter estimation
            \item Active learning for minimal data collection
          \end{itemize}
          
          \item \textbf{Scaling the Framework}
          \begin{itemize}
            \tiny
            \item Sparse/inducing-point GPs $\Rightarrow$ $\mathcal{O}(N \log N)$
            \item Extension to $10^2$-$10^3$ parameters
            \item Real-time implementation strategies
          \end{itemize}
        \end{itemize}
      \end{block}
      
      \vspace{0.5em}
      % Software availability
      \begin{center}
        \tiny
        \textbf{Open Source Software:}\\
        \texttt{github.com/orebas/ODEParameterEstimation}
        
        \vspace{0.3em}
        % TODO: Add QR code here
        \fbox{
          \begin{minipage}{0.6\textwidth}
            \centering
            \textit{TODO: QR code}
          \end{minipage}
        }
      \end{center}
    \end{column}
    
  \end{columns}
  
  \vspace{1em}
  \begin{alertblock}{Overall Impact}
    \centering
    \footnotesize
    \textbf{GPR transforms the algebraic parameter estimation method from a theoretical curiosity} \\
    \textbf{into a robust, practical tool for real-world system identification.}
  \end{alertblock}
  
\end{frame}

% BACKUP/APPENDIX SLIDES FOR Q&A

% Appendix A1: Computational Performance Analysis
\begin{frame}{A1: Computational Performance Analysis}
  
  \begin{columns}[t]
    \begin{column}{0.48\textwidth}
      \begin{block}{\small Timing Benchmarks}
        \tiny
        \textbf{Runtime Comparison (10k samples):}
        \begin{itemize}
          \item GPR\_Julia: 2.15s
          \item SavitzkyGolay: 0.0015s  
          \item TVDiff: 0.19s
          \item FiniteDiff: 0.0006s
        \end{itemize}
        
        \vspace{0.5em}
        \textbf{Runtime Breakdown (GPR):}
        \begin{itemize}
          \item Polynomial solving: 70\%
          \item GPR computation: 20\%
          \item Miscellaneous: 10\%
        \end{itemize}
      \end{block}
      
      \vspace{0.5em}
      \begin{block}{\small Scaling Analysis}
        \tiny
        \textbf{Computational Complexity:}
        \begin{itemize}
          \item Naive GP: $\mathcal{O}(N^3)$
          \item Sparse GP: $\mathcal{O}(NM^2)$, $M \ll N$
          \item Current bottleneck: Polynomial solving
        \end{itemize}
      \end{block}
    \end{column}
    
    \begin{column}{0.48\textwidth}
      % Placeholder for timing chart
      \begin{center}
        \fbox{
          \begin{minipage}{0.9\textwidth}
            \centering
            \textit{TODO: Bar chart showing}\\
            \textit{runtime breakdown and}\\
            \textit{method comparison}
          \end{minipage}
        }
      \end{center}
      
      \vspace{0.5em}
      \begin{alertblock}{\small Key Insight}
        \footnotesize
        GPR is \textbf{not} the computational bottleneck.\\
        Accuracy-first approach justified by\\
        minimal overhead vs. dramatic robustness gains.
      \end{alertblock}
    \end{column}
  \end{columns}
  
\end{frame}

% Appendix A2: Extended Benchmark Results
\begin{frame}{A2: Extended Benchmark Results}
  
  \begin{center}
    \small
    \textbf{Complete Performance Matrix: 21 Methods $\times$ 3 Noise Levels $\times$ 6 Derivative Orders}
  \end{center}
  
  \vspace{0.5em}
  % Placeholder for comprehensive heatmap
  \begin{center}
    \fbox{
      \begin{minipage}{0.8\textwidth}
        \centering
        \textit{TODO: Comprehensive heatmap showing}\\
        \textit{$\log_{10}$(RMSE) across all conditions}\\
        \textit{Red = failure, Green = success}\\
        \textit{Generated by method\_comparison\_heatmap.pdf}
      \end{minipage}
    }
  \end{center}
  
  \vspace{0.5em}
  \begin{block}{\small Statistical Analysis}
    \tiny
    \begin{itemize}
      \item \textbf{Failure Rate Analysis:} GPR\_Julia: 0\% (0/63), AAA\_Julia: 37\% (23/63), TVDiff: 52\% (33/63)
      \item \textbf{Edge Cases:} Very low noise ($\sigma < 10^{-10}$) requires numerical conditioning
      \item \textbf{Confidence Intervals:} 95\% CI on geometric mean RMSE across 30 MC runs per condition
      \item \textbf{Method Categories:} Finite differences, spectral, variational, kernel-based
    \end{itemize}
  \end{block}
  
\end{frame}

% Appendix A3: GPR Implementation Details
\begin{frame}{A3: GPR Implementation \& Methodology Details}
  
  \begin{columns}[t]
    \begin{column}{0.48\textwidth}
      \begin{block}{\small Hyperparameter Optimization}
        \tiny
        \textbf{Marginal Likelihood Maximization:}
        $$\log p(y|\theta) = -\frac{1}{2}y^T\Sigma^{-1}y - \frac{1}{2}\log|\Sigma| - \frac{n}{2}\log 2\pi$$
        
        \textbf{Optimization Strategy:}
        \begin{itemize}
          \item L-BFGS-B with multiple random initializations
          \item Parameter bounds: $\ell \in [10^{-3}, 10^3]$, $\sigma_f^2 \in [10^{-6}, 10^6]$
          \item Automatic noise floor: $\sigma_n^2 \geq 10^{-12}$
        \end{itemize}
      \end{block}
      
      \vspace{0.5em}
      \begin{block}{\small Kernel Selection}
        \tiny
        \textbf{RBF vs. Matérn Comparison:}
        \begin{itemize}
          \item RBF: $C^{\infty}$ smooth, best for high-order derivatives
          \item Matérn-3/2: $C^1$ smooth, computational efficiency
          \item Matérn-5/2: $C^2$ smooth, good compromise
        \end{itemize}
        
        \textbf{Empirical Results:} RBF kernel optimal for derivative orders $\geq 3$
      \end{block}
    \end{column}
    
    \begin{column}{0.48\textwidth}
      \begin{block}{\small Integration Framework}
        \tiny
        \textbf{Automatic Differentiation:}
        \begin{itemize}
          \item Julia: ForwardDiff.jl for exact derivatives
          \item Python: JAX for vectorized operations
          \item Enables fair, consistent method comparison
        \end{itemize}
        
        \textbf{Software Integration:}
        \begin{itemize}
          \item Modular design: derivative estimator + algebraic solver
          \item Compatible with existing parameter estimation pipelines
          \item Export to MATLAB, Python, Julia formats
        \end{itemize}
      \end{block}
      
      \vspace{0.5em}
      % Placeholder for methodology diagram
      \begin{center}
        \fbox{
          \begin{minipage}{0.9\textwidth}
            \centering
            \textit{TODO: Workflow diagram}\\
            \textit{Data → GPR → Derivatives → }\\
            \textit{Algebraic Solver → Parameters}
          \end{minipage}
        }
      \end{center}
    \end{column}
  \end{columns}
  
\end{frame}

\end{document}